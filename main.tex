\documentclass[a4paper,12pt]{article}

\usepackage[top = 2cm, bottom = 2cm, left = 2cm, right = 2cm]{geometry} 

\usepackage[T1]{fontenc}
\usepackage[utf8]{inputenc}
\usepackage[spanish]{babel}
\usepackage{multirow}
\usepackage{booktabs} 
\usepackage{graphicx} 
\usepackage{setspace}
\setlength{\parindent}{0in}
\usepackage{float}
\usepackage{fancyhdr}
\usepackage{enumitem}
\usepackage{tikz-timing}
\usepackage{amsmath,amssymb,amsfonts}
\usepackage[thinlines]{easytable}
\usepackage{array}
\usepackage{listings}
\usepackage{hyperref} % Support for hyperlinks
\usepackage{xcolor}

\definecolor{LightGray}{gray}{0.95}

\definecolor{mGreen}{rgb}{0,0.4,0}

\lstset{
    frame=tblr, % draw a frame at the top and bottom of the code block
    tabsize=4, % tab space width
    showstringspaces=false, % don't mark spaces in strings
    numbers=left, % display line numbers on the left
    commentstyle=\color{mGreen}, % comment color
    keywordstyle=\color{blue}, % keyword color
    stringstyle=\color{red} % string color
}


\tikzset{
timing/z/.style={color=red},
timing/l/.style={color=red},
timing/h/.style={color=red},
timing/slope=0,
timing/name/.append style={yshift=1.5mm}
}

\pagestyle{fancy} 
\lhead{\footnotesize Construcción de Software: Laboratorio 8}

\rhead{\footnotesize 30 de noviembre del 2020}

\cfoot{\footnotesize \thepage} 

\begin{document}

\thispagestyle{empty}

\begin{tabular}{p{15.5cm}} 
\large Universidad Nacional de San Agustín \\ 
\large Facultad de Ingeniería de Producción y Servicios \\
\large Escuela Profesional de Ingeniería de Sistemas \\
{\LARGE \bf Construcción de Software} \\
\vspace{1mm}
Alumno: Mestas Escarcena, Carlos Alberto \\
Grupo: B \\
\hline \\
\end{tabular} 

\begin{center} 
	{\LARGE \bf Laboratorio 9 - UML aplicado}
	\vspace{2mm}
\end{center}  

\section*{Ejercicio 1}

\begin{itemize}
    \item 11. El 16 de noviembre Francisco Sagasti es elegido como presidente del congreso de la República del Perú.
    \item 12. El 17 de noviembre juramenta como presidente constitucional del Perú. La ONU felicita a Francisco Sagasti como presidente del Perú.
    \item 13. El 18 de noviembre distintos congresistas se pronuncian sobre el fallecimiento de los jóvenes en las protestas
\end{itemize}

\section*{Ejercicio 2}



\clearpage
\newpage

El código utilizado para generar el diagrama fue:

\begin{lstlisting}
@startuml


node "Servidor de Aplicaciones" as sa{
    component "Capa de\nnegocio" as c01
    component "Capa de\nacceso a\ndatos" as c02
}

database "Base de Datos" as bd{
    component "SQL Server" as sql01
}

node "router"

node "PC secretaria" as pc1 {
    component "Capa de\npresentacion" as c03
}

node "PC doctor" as pc2 {
    component "Capa de\npresentacion" as c04
}

node "PC asistente" as pc3 {
    component "Capa de\npresentacion" as c05
}

node "HP Laserjet Pro M402dn" as hp

router -- pc1 : LAN
router -- pc2 : LAN
router -- pc3 : LAN

router -left- hp : LAN

sa -- bd

pc1 -- sa
pc2 -- sa
pc3 -- sa

@enduml
\end{lstlisting}

\begin{figure}[ht]
        \centering        
        \includegraphics[width=1\textwidth]{images/cs01.png}
        \caption{Diagrama de despliegue - Ejercicio 1.}  
        {{\footnotesize Realizado en: PlantText UML Editor }}
\end{figure}

\section*{Ejercicio 2}

El código utilizado para generar el diagrama fue:

\begin{lstlisting}
@startuml

node "Servidor Web" as sw

node "Servidor de Información" as si {
    component "Interfaces" as i1
    component "Habitaciones" as h1
}

node "Clientes" as c

node "PC recepcionista" as pc0

node "Servidor Espejo" as se {
    component "Interfaces" as i2
    component "Habitaciones" as h2
}

node "Router" as r

node "PC 1" as pc1

node "PC 2" as pc2

node "PC 3" as pc3

node "PC 4" as pc4

node "PC 5" as pc5

node "Impresora" as i

r -- i      : LAN
r -- pc1    : LAN
r -- pc2    : LAN
r -- pc3    : LAN
r -- pc4    : LAN
r -- pc5    : LAN

se "1" -- "1" si

si -- pc0

pc0 -- r : LAN

sw -right- si

c -- sw : TCP / IP

@enduml
\end{lstlisting}

\clearpage
\newpage

\begin{figure}[ht]
        \centering        
        \includegraphics[width=1\textwidth]{images/cs02.png}
        \caption{Diagrama de despliegue - Ejercicio 2.}  
        {{\footnotesize Realizado en: PlantText UML Editor }}
\end{figure}

\section*{Ejercicio 3}

El código utilizado para generar el diagrama fue:

\begin{lstlisting}
@startuml


node "Router" as r

node "PC 1" as pc1{
    component "Pago efectivo" as p1
    component "Ventas" as v1
}

node "PC 2" as pc2 {
    component "Pago POS Visa" as p2
    component "Ventas" as v2
}

node "PC 3" as pc3{
    component "Pago POS Mastercard" as p3
    component "Ventas" as v3
}

node "Almacen" as alm {
    node "PC 4" as pc4{
        component "Inventario" as p4
    }
}

node "Servidor de aplicaciones" as ser 

database "Base de Datos" as bd{
    component "SQL Server" as sql01
}

node "Impresora" as i

r -up- pc1    : LAN
r -up- pc2    : LAN
r -up- pc3    : LAN
r -- i      : LAN

ser -left- bd

r -- ser
ser -- pc4

@enduml
\end{lstlisting}

\clearpage
\newpage

\begin{figure}[ht]
        \centering        
        \includegraphics[width=1\textwidth]{images/cs03.png}
        \caption{Diagrama de despliegue - Ejercicio 3.}  
        {{\footnotesize Realizado en: PlantText UML Editor }}
\end{figure}

\section*{Ejercicio 4}

El código utilizado para generar el diagrama fue:

\begin{lstlisting}
@startuml

node "Router" as r

node "PC 1" as pc1{
    component "Pago efectivo" as p1
    component "Ventas" as v1
}

node "PC 2" as pc2 {
    component "Pago POS Visa" as p2
    component "Ventas" as v2
}

node "PC 3" as pc3{
    component "Pago POS Mastercard" as p3
    component "Ventas" as v3
}

node "Almacen" as alm {
    node "PC 4" as pc4{
        component "Inventario" as inv
        component "Despacho" as des
    }
}

node "Servidor de aplicaciones" as ser {
    
    component "Ingreso" as ing
    component "Menu" as menu
    component "Catalogos" as cat
}

node "Servidor web" as sw {
    component "Aplicacion" as app
    component "Apache Tomcat" as at
    component "Utilidades" as u
}

database "Base de Datos" as bd{
    component "SQL Server" as sql01
}

node "Impresora" as i

r -up- pc1    : LAN
r -up- pc2    : LAN
r -up- pc3    : LAN
r -- i      : LAN

ser -left- bd

r -- ser
ser -- pc4
ser -- sw

v1 ..> ing
v2 ..> ing
v3 ..> ing

menu ..> ing
cat ..> ing

des ..> ing
inv --> ing

@enduml
\end{lstlisting}

\clearpage
\newpage

\begin{figure}[ht]
        \centering        
        \includegraphics[width=1\textwidth]{images/cs04.png}
        \caption{Diagrama de despliegue - Ejercicio 4.}  
        {{\footnotesize Realizado en: PlantText UML Editor }}
\end{figure}

\section*{Ejercicio 5}

Para el diagrama del ejercicio he planteado las siguientes consideraciones:

\begin{itemize}
    \item Diseñar el diagrama de despliegue de una agencia de turismo de la plaza de armas de la ciudad de Arequipa.
    \item Se tienen dos computadoras y una impresora conectadas a una red local.
    \item Se tiene un servidor de aplicaciones al cuál podrán ingresar desde estas computadoras, para el uso de las aplicaciones del menú, ventas y catálogos se requiere un previo ingreso al sistema.
    \item Se tiene también una impresora en la red local para poder imprimir diferentes tickets.
    \item Se tiene también un servidor web desde el cuál los turistas podrán obtener información de la agencia y que ofrece.
    
\end{itemize}

\clearpage
\newpage

El código utilizado para generar el diagrama fue:

\begin{lstlisting}
@startuml

node "Router" as r

node "PC 1" as pc1{

}

node "PC 2" as pc2{

}


node "Servidor de aplicaciones" as ser {
    component "Ingreso" as ing
    component "Menu" as menu
    component "Catalogos" as cat
    component "Ventas" as ven
}

node "Servidor web" as sw {
    component "Aplicacion" as app
    component "Apache Tomcat" as at
    component "Utilidades" as u
}

database "Base de Datos" as bd{
    component "SQL Server" as sql01
}

node "Impresora" as i

pc1 -- r : LAN
pc2 -- r : LAN
i -- r : LAN

ser -- r
sw -- ser
bd -- ser

menu ..> ing
cat ..> ing
ven ..> ing


@enduml
\end{lstlisting}

\begin{figure}[ht]
        \centering        
        \includegraphics[width=1\textwidth]{images/cs05.png}
        \caption{Diagrama de despliegue - Ejercicio 5.}  
        {{\footnotesize Realizado en: PlantText UML Editor }}
\end{figure}

\clearpage
\newpage





    
\end{document}
